\documentclass[10pt]{beamer}


% colors
\usepackage{xcolor}
\definecolor{mydarkgray}{gray}{0.33}
\definecolor{myred}{rgb}{0.85, 0.30, 0.0}
\definecolor{myblue}{rgb}{0.01, 0.45, 0.70}
\definecolor{mylightgray}{gray}{0.5}
\definecolor{mylightestgray}{gray}{0.9}

% beamer options
\setbeamertemplate{caption}[numbered]
\setbeamertemplate{frametitle}[default][center]
\setbeamertemplate{itemize items}[circle]
\setbeamertemplate{itemize subitem}{$\circ$}

\setbeamercolor{title}{fg=myblue}
\setbeamercolor{titlelike}{fg=mydarkgray}
\setbeamercolor{frametitle}{fg=mydarkgray, bg=mylightestgray}
\setbeamercolor{framesubtitle}{fg=mylightgray}
\setbeamercolor{itemize item}{fg=mydarkgray}
\setbeamercolor{itemize subitem}{fg=mydarkgray}

% tables
\usepackage{booktabs} 

% figures
\usepackage{graphicx}
\usepackage{float}
\graphicspath{ {/Users/gexuanzhe/Desktop/Digital_Tools_For_Finance} }

% language
\usepackage[english]{babel}
\usepackage[utf8]{inputenc}
\usepackage[T1]{fontenc}

% titlepage
\author{Alain Gadient, Hanxiao Qu, Mann-Tchi Dang, Xuanzhe Ge}

\title{Which of the G10 Currencies is the Riskiest to Hold for a Swiss Resident?}

\institute{University of Zurich}
\date{\today}

\begin{document}
% ---------------------------------------------------------------------------
\begin{frame}
    \titlepage
\end{frame}
% ---------------------------------------------------------------------------
\begin{frame}
    \frametitle{Outline}
    \tableofcontents
\end{frame}
% ---------------------------------------------------------------------------
\begin{frame}
\section{Introduction}
G10 currencies:\\ 
the US dollar (USD), the euro (EUR)\\ 
the Japanese yen (JPY), the UK pound sterling (GBP),
\\the Swiss franc (CHF), the Norwegian krone (NOK), 
\\the Swedish krona (SEK), the Canadian dollar (CAD), 
\\the Australian dollar (AUD) and the New Zealand dollar (NZD)
\frametitle{Introduction}
\framesubtitle{Components of G10 currency}

\end{frame}
% ---------------------------------------------------------------------------
\begin{frame}
Currency Pairs for Swiss Citizen: \\ 
CHF/USD, CHF/EUR, CHF/GBP, CHF/JPY,\\ 
CHF/AUD, CHF/NZD, CHF/CAD, CHF/NOK, \\
CHF/SEK
\\ \hspace*{\fill} \\
Our goal is to analysis these pairs returns and find the riskiest one
\frametitle{Introduction}
\framesubtitle{Currency Pairs}

\end{frame}
% ---------------------------------------------------------------------------
\begin{frame}
\section{Data}
Daily data of the exchange rate of currency of one year from 2020 to 2022 is used
\\ \hspace*{\fill} \\
The data is retrieved  from the API of apilayer as one of the known sources for apis
\\ \hspace*{\fill} \\
Some spot exchange rate data is retrieved from several major forex data providers in real-time, validated, processed and delivered hourly, every 10 minutes, or even within the 60-second market window.
\frametitle{Data}

\end{frame}
% ---------------------------------------------------------------------------
\begin{frame}
\section{Model}
Sample Variance is used to estimate the risk. the formula is as follows:

\begin{align}
	s^2 =  \frac{\sum_{i}^{N}{(X-\mu)^2}}{N-1} 
\end{align}

Expected shortfall is expressed as follows:

\begin{align}
ES_{\alpha}(X) = -\frac{1}{\alpha}\int_{0}^{\alpha} VaR_{\gamma}(X)d\gamma\\
VaR_{\alpha}(X) = - inf \{x \in R:F_{X}(X)>\alpha \}
\end{align}

\frametitle{Model}

\end{frame}
% ---------------------------------------------------------------------------
\begin{frame}
\section{Empirical Results}

\begin{table}\small
  \resizebox{11cm}{!}{
   \begin{tabular}{cccccccccc}
   \toprule
             &CHFUSD & CHFEUR & CHFGBP & CHFJPY & CHFAUD & CHFNZD & CHFCAD & CHFNOK & CHFSEK\\
   \midrule
   Mean & 0.000228 & 0.000005 & 0.000142 & 0.000087 & -0.000022 & 0.000294 & 0.000178 & 0.000184 & -0.000106\\
   Median & 0.000228 & 0.000005 & 0.000142 & 0.000087 & -0.000022 & 0.000294 & 0.000178 & 0.000184 & -0.000106\\
   Std. dev. & 0.000228 & 0.000005 & 0.000142 & 0.000087 & -0.000022 & 0.000294 & 0.000178 & 0.000184 & -0.000106\\
   Std. dev. & 0.000228 & 0.000005 & 0.000142 & 0.000087 & -0.000022 & 0.000294 & 0.000178 & 0.000184 & -0.000106\\
   \bottomrule
   \end{tabular}}
   \caption{Summary Table of the Returns}
\end{table}

\begin{figure}[H]
\centering
\includegraphics[scale = 0.4]{Return}
\caption{Return Histogram}\label{visina8}
\end{figure}
\end{frame}
% ---------------------------------------------------------------------------
\begin{frame}
\frametitle{Empirical Results}
\framesubtitle{Return Summary Table}

On the heat map of ES (figure 2) we can see that CHF/NOK has the highest downside Risk among all the other currency pairs.

\begin{figure}[H]
\centering
\includegraphics[scale = 0.3]{ES}
\caption{Expected Shortfall}\label{visina8}
\end{figure}

\frametitle{Empirical Results}
\framesubtitle{Expected Shortfall Heat Map}
\end{frame}
% ---------------------------------------------------------------------------
\begin{frame}

On the heat map of Variances (figure 3) we can also notice that CHF/NOK is attached to the highest variance among all the other Currency pairs.

\begin{figure}[H]
\centering
\includegraphics[scale = 0.3]{Variance}
\caption{Variance}\label{visina8}
\end{figure}

\frametitle{Empirical Results}
\framesubtitle{Variance Heat Map}
\end{frame}

% ---------------------------------------------------------------------------
\begin{frame}
\section{Conclusion}

Both of this riskmeasures has shown that CHF/NOK is the riskiest over the year of 2021. Therefore, during the 2021, we come to a conlusion that CHF/NOK is the riskiest Currency pair position to hold
\\ \hspace*{\fill} \\
In another words, to hold the he Norwegian krone (NOK) is the riskiest strategy for a Swiss citizen.

\begin{figure}[H]
\centering
\includegraphics[scale = 0.25]{Interactive}
\caption{Interactive app}\label{visina8}
\end{figure}

\frametitle{Conclusion}

\end{frame}
% ---------------------------------------------------------------------------

\end{document}